\documentclass{beamer}

\usepackage[utf8]{inputenc}
\usepackage[russian]{babel}
\usepackage{verbatim}

\title{Операционные системы и сети}
\subtitle{Введение в ОС}
\author{Филонов~Павел}
\institute{filonovpv@gmail.com}

\usetheme{Warsaw}

\begin{document}
\titlepage
\begin{frame}
    \frametitle{Введение в операционные системы}
    {\bfОперационная система}, сокр. ОС (Operating System, OS):
    \begin{itemize}
        \item совокупность 
    системных программ, предназначенная для обеспечения определенного уровня 
    эффективности системы обработки информации за счет автоматизированного 
    управления ее работой и предоставляемого пользователю определенного набора
    услуг\footnote{ГОСТ 15971-90 Системы обработки информации. Термины и определения};
        \item программное обеспечение, которое позволяет приложениям взаимодействовать
    аппаратными средствами компьютера\footnote{Х.~М.~Дейтел, П.~Дж.~Дейтел, Д.~Р.~Чофнес
    Операционные системы. Основы и принципы}.
    \end{itemize}
    С позиции разработчиков программного обеспечения ОС представляет собой виртуальную
    машину, которая предоставляет доступ к абстрактным услугам и ресурсам (память, файлы, сетевое взаимодействие и т.д.)
\end{frame}

\begin{frame}
    \frametitle{Функции современных ОС}
    \begin{itemize}
        \item мультизадачный режим работы;
        \item управление устройствами ввода-вывода;
        \item управление оперативной памятью;
        \item взаимодействие процессов;
        \item разграничение полномочий.
    \end{itemize}
\end{frame}

\begin{frame}
    \frametitle{Понятие процесса}
    {\bf Программа} --- исполняемый файл

    {\bf Процесс} --- экземпляр программа, которая исполняется. Включает в себя
    машинный код и текущее состояние (значения регистров, оперативной памяти и проч.)

    {\bf Идентификатор процесса} (process identifier, PID) --- целое, число которое ОС
    выдает при запуске нового процесса. Процесс идентифицируется именно по данному числу.
    Обычно идентификаторы представляют собой числа от 0 до $2^16 - 1= 65535$.

    {\bf Планировщик процессов} (process scheduler) --- часть ОС, которая отвечает за
    распределение ресурсов процессора между задачами.

    {\bf Поток исполнения} (thread of execution) --- минимальный набор машинных инструкций, который может независимо управляться планировщиком.
\end{frame}

\begin{frame}[fragile]
    \frametitle{Пример: PID --- идентификатор процесса}
    Команда {\bf ps ax} отображает список всех запущенных процессов.

    Команда {\bf pstree} отображает дерево процессов.

    Системный вызов {\bf getpid} возвращает идентификатор процесса.
    \verbatiminput{src/printpid.c}
\end{frame}


\begin{frame}
    \frametitle{Машина состояний процесса}
        \includegraphics[width=\linewidth]{fig/process_state_machine-crop.pdf}
\end{frame}

\begin{frame}[fragile]
    \frametitle{Пример: создание процесса}
    Системный вызов {\bf fork} порождает копию процесса, которая начинает выполнение
    со следующей после этого вызова инструкции. В родительский процесс возвращается идентификтор созданного дочернего процесса, в в дочернем процессе возвращается ноль.

    Системный вызов {\bf wait} ожидает завершения дочернего процесса и возвращает его номер либо код ошибки (-1).
\begin{verbatim}
pid_t child_pid = fork();
if (!child_pid)  {
    /* Child branch. Usually ends with exit*/
} 
/* Parent branch. */
int rv = wait(NULL);
\end{verbatim}
\end{frame}

\begin{frame}[fragile]
    \frametitle{Пример: запуск исполняемого файла. Замещение процесса.}
    Семейство системных вызовов {\bf exec} замещает текущий процесс новым из
    исполняемого файла, указанного в параметрах.

    Системный вызов {\bf exit()} завершает выполнение процесса и запусывает его код завершения. Код отличный от нуля сигнализирует об ошибке.
\begin{verbatim}
pid_t child_pid = fork();
if (!child_pid)  {
    execlp("ls", "ls", NULL);
    exit(1);
} 
/* Parent branch. */
int rv = wait(NULL);
\end{verbatim}

\end{frame}
% \begin{frame}
%     \frametitle{Аппаратные прерывания процессора}
%     \begin{enumerate}
%         \item устройство, устанавливает на шине данных сигнал о прерывании;
%         \item процессор устанавл
%     \end{enumerate}
% \end{frame}
\begin{frame}
    \frametitle{Стратегии планирования процессов}
    \begin{itemize}
        \item Первый пришел - первй обслуживается. Fist Come - First Served (FCFS)
        \item Наиболее короткая задача. Shortest Job First (SJF)
        \item Приоритетное планирование
        \item <<Карусельное>> планирование. Round Robin (RR)
        \item Многоуровневые очереди. Multilevel Queue Scheduling
        \item Многоуровневые очереди с обратной связью. Multilevel Feddback Queue Scheduling.
        \item <<Честное>> планирование. Fair scheduling
    \end{itemize}
\end{frame}
\end{document}
