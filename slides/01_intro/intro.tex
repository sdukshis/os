\documentclass{beamer}

\usepackage[utf8]{inputenc}
\usepackage[russian]{babel}

\title{Операционные системы и сети}
\subtitle{Введение в ОС}
\author{Филонов~Павел}
\institute{filonovpv@gmail.com}

\usetheme{Warsaw}

\begin{document}
\titlepage
\begin{frame}
    \frametitle{Введение в операционные системы}
    {\bfОперационная система}, сокр. ОС (Operating System, OS):
    \begin{itemize}
        \item совокупность 
    системных программ, предназначенная для обеспечения определенного уровня 
    эффективности системы обработки информации за счет автоматизированного 
    управления ее работой и предоставляемого пользователю определенного набора
    услуг\footnote{ГОСТ 15971-90 Системы обработки информации. Термины и определения};
        \item программное обеспечение, которое позволяет приложениям взаимодействовать
    аппаратными средствами компьютера\footnote{Х.~М.~Дейтел, П.~Дж.~Дейтел, Д.~Р.~Чофнес
    Операционные системы. Основы и принципы}.
    \end{itemize}
    С позиции разработчиков программного обеспечения ОС представляет собой виртуальную
    машину, которая предоставляет доступ к абстрактным услугам и ресурсам (память, файлы, сетевое взаимодействие и т.д.)
\end{frame}

\begin{frame}
    \frametitle{Функции современных ОС}
    \begin{itemize}
        \item мультизадачный режим работы;
        \item управление устройствами ввода-вывода;
        \item управление оперативной памятью;
        \item взаимодействие процессов;
        \item разграничение полномочий.
    \end{itemize}
\end{frame}

\begin{frame}
    \frametitle{Понятие процесса}
    {\bf Программа} --- исполняемый файл

    {\bf Процесс} --- экземпляр программа, которая исполняется. Включает в себя
    машинный код и текущее состояние (значения регистров, оперативной памяти и проч.)

    {\bf Идентификатор процесса} (process identifier, PID) --- целое, число которое ОС
    выдает при запуске нового процесса. Процесс идентифицируется именно по данному числу.
    Обычно идентификаторы представляют собой числа от 0 до $2^16 - 1= 65535$.

    {\bf Планировщик процессов} (process scheduler) --- часть ОС, которая отвечает за
    распределение ресурсов процессора между задачами.

    {\bf Поток исполнения} (thread of execution) --- минимальный набор машинных инструкций, который может независимо управляться планировщиком.
\end{frame}

\begin{frame}
    \frametitle{Системные вызовы для создания процессов POSIX}

\end{frame}
\end{document}
