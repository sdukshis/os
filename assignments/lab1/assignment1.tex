\documentclass{extarticle}

\usepackage[utf8]{inputenc}
\usepackage[russian]{babel}
\usepackage{cmap}
\usepackage{hyperref}
\usepackage{cite}

\title{Задание №1. \\ Создание и завершение процессов}

\begin{document}
\maketitle

Каждый раз при открытии окна терминала в любой операционной системе запускается специальная программа --- командный интерпретатор (оболочка). Например, для Windows это будет {\tt cmd.exe}, для Linux и Mac OS X --- {\tt bash}. После запуска оболочка выводит приглашение командной строки (например символ {\tt \$}) и ожидает ввода команды с клавиатуры. Команда обычно включает в себя имя исполняемого файла, который следует запустить и аргументов командной строки, которые нужно этой программе передать. 

Пример сеанса работы с оболочкой:
\begin{verbatim}
$ls
assignment1.aux assignment1.log assignment1.pdf assignment1.tex
$cal
    Марта 2016
вс пн вт ср чт пт сб
       1  2  3  4  5
 6  7  8  9 10 11 12
13 14 15 16 17 18 19
20 21 22 23 24 25 26
27 28 29 30 31
$nasm -f elf assignment.asm
$ld -o assignment assignment.o
\end{verbatim}

После запуска на выполнение каждой очередной команды оболочка ждет пока 
команда отработает, выводит новое приглашение командной строки и ожидает 
ввода новой команды.

{\bf Задание:} написать программу на языке C\cite{KnR} c использование POSIX\cite{Stolyarov} системных вызовов, которая реализует базовый функционал оболочки. Программа должна в бесконечном цикле считывать команды пользователя и запускать соответствующие им программы. Завершать работу следует при возникновении ситуации <<конец файла>>.

{\bf Указания:} в ходе работы потребуется использовать системные вызовы {\tt fork, execlp, wait}. Ознакомьтесь с их описанием с помощью справочного руководства {\tt man}.
\begin{verbatim}
$man 2 fork
\end{verbatim}

Для запуска исполняемого файла с помощью {\tt execvp} необходимо подготовить 
структуру данных в виде массива строк, в которую записать имя программы и ее аргументы. Предлагается ввести ограничение на число аргументов в 16 и на максимальную длину каждого аргумента в 80 символов. Тогда эту структуру 
можно представить как двумерный символьный массив фиксированного размера.
\begin{verbatim}
char argv[16][80];
\end{verbatim}

Запуск внешней программы в POSIX реализуется с помощью связки {\tt fork+exec}.
Ниже приведен пример запуска программы {\tt ls} c аргументами {\tt -l} и {\tt -a}.
\begin{verbatim}
/* Exec ls -l -a */

#include <stdio.h>
#include <stdlib.h>
#include <sys/types.h>
#include <sys/wait.h>
#include <unistd.h>

int main()
{
  pid_t pid = fork();
  if (!pid) { // child branch
    int rv = execlp("ls", "ls", "-l", "-a", NULL);
    if (rv == -1) {
      perror("execlp");
      return EXIT_FAILURE;
    }
  }
  // parent branch
  pid = wait(NULL);
  if (pid == -1) {
    perror("wait");
    return EXIT_FAILURE;
  }
  return EXIT_SUCCESS;
}
\end{verbatim}

Результаты данной работы станут основой для выполнения дальнейших заданий.
В связи с этим предлагается использовать для разработки систему контроля версий (например {\tt git}\cite{GitCookbook}) и хранить результаты работ в удаленном репозитории (например {\tt github.com}).

\begin{itemize}
    \item Создайте учетную записть на \url{http://github.com}.
    \item Создайте новый репозиторий (например {\tt OSLabs}).
    \item Добавьте в репозиторий код решения.
    \item Пришлите ссылку на репозиторий для проверки.
\end{itemize}

\begin{thebibliography}{9}
    \bibitem{KnR} Б. Керниган, Д. Ритчи. Язык программирования C, 2-е изд. — Москва: Вильямс, 2015. — 288 с. — ISBN 978-5-8459-1976-5.
    \bibitem{Stolyarov} А. Столяров. Введение в операционные системы, М.: Изд. отдел ВМК МГУ, 2006. - ISBN 5-89407-246-8. \url{http://stolyarov.info/books/osintro}
    \bibitem{GitCookbook} \url{https://git-scm.com/book/ru/v1}
\end{thebibliography}
\end{document}
